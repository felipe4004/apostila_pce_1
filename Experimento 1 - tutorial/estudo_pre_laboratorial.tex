\section{Estudo pré-laboratorial}

Nessa secção dar-se-á uma explicação, no sentido lato, sobre as características principais dos demais instrumentos de bancada.

\subsection{Multímetro}

Quando mede-se uma tensão AC com multímetro, há interesse em saber o valor eficaz do sinal, ou seja, o valor RMS. Entretanto, há apenas uma classe específica de multímetros que podem fazer tal medição: os True RMS. Multímetros classificados com True RMS permitem a leitura precisa mesmo para tensões AC muito distorcidas.

É importante ressaltar que o conceito de sinal AC refere-se às correntes alternadas geralmente no formato senoidal, mas a definição também abarca as ondas triangulares ou quadradas. A corrente alternada é, portanto, toda corrente elétrica cujo sentido varia no tempo, ao contrário da corrente contínua cujo sentido permanece constante ao longo do tempo.

As formas dos sinais AC para leitura via multímetro dependem, em grande parte, da frequência em que a corrente varia. Os multímetros conseguem captar uma boa leitura em uma faixa de valores específica para cada modelo. A extrapolação desses valores implicam em uma leitura com pouca acurácia e, em casos de leituras com baixa frequência, um erro de precisão.

Use, agora, os manuais dos equipamentos, as referências bibliográficas e demais fontes de informação para responder as seguintes questões:

\vspace{.3cm}

a) Sobre o multímetro ET-1110 da Minipa:
\begin{itemize}
    \item Como devem ser os sinais AC para que possam ser medidos com o multímetro?
    \item Quais as medidas de um sinal AC que podemos obter com o multímetro? O que é peculiar a um multímetro "true rms"?
\end{itemize} 


\subsection{Geradores de funções}

Os geradores de funções trabalham gerando algumas formas de onda, geralmente triangular, quadrada e senoidal. O equipamento permite manipular a amplitude, frequência, duty cicle e o offset dos sinais. As faixas de operação do equipamento dependem do modelo e das condições de uso.

Use, agora, os manuais dos equipamentos, as referências bibliográficas e demais fontes de informação para responder as seguintes questões:

\vspace{.3cm}

b) Sobre os geradores de funções: modelo SDG 1020 da SIGLENT e modelo GV-2002 da iCEL:
\begin{itemize}
    \item Quais as formas de onda possíveis de serem geradas pelo gerador de funções?
    \item Qual o intervalo de frequências permitido pelo equipamento?
    \item Qual a amplitude máxima e mínima possíveis para as formas de onda?
    \item Esse gerador de funções produz valor DC de tensão? Se sim, como? Se não, por quê?
\end{itemize}

\subsection{Osciloscópio}

Se ao gerador de funções é dada a função de produzir formas de onda variantes no tempo, ao osciloscópio é dada a função de análise desses sinais, não so gerados pelo gerador de funções, mas de quaisquer espécie desde que seja periódico e com um padrão definido. O osciloscópio é um aparelho eletrônico que permite-se visualizar e analisar uma diferença de potencial (DDP) em função do tempo em um gráfico bidimensional.

Atualmente os osciloscópios contam com uma gama de operações matemáticas, as quais permitem somar, subtrair, multiplicar e dividir sinais, além de poder aplicar transformações de domínio como a FFT ("\textit{Fast Fourier Transform}").

Use, agora, os manuais dos equipamentos, as referências bibliográficas e demais fontes de informação para responder as seguintes questões:

\newpage

c) Osciloscópio modelo 2530 da BK Precision:
\begin{itemize}
    \item Explique a funcionalidade da função MEASURE.
    \item Explique como medir a amplitude e a frequência de um sinal periódico no osciloscópio sem o auxílio da função MEASURE (Dica: procure sobre a ferramenta "\textit{cursors}"). 
    \item 
\end{itemize}

\subsection{Pesquise e responda:}

\noindent b) O que é o valor pico-a-pico de um sinal AC? E a amplitude?

\noindent c) O que é o valor médio (também chamado de valor DC) de um sinal AC? Como ele pode ser calculado?

\noindent d) O que é o valor eficaz (também chamado de valor RMS) de um sinal AC? Como ele pode ser calculado?

\noindent e) Esboce as formas de onda e calcule a tensão eficaz $V_ef$ para os seguintes sinais:

\begin{table}[H]
    \centering
    \begin{tabular}{|c|c|c|c|l|}
    \hline
    Formas de onda  & Frequência (kHz) & Valor Médio (V) & Amplitude (V) & Valor Eficaz (V) \\
    \hline
    C1 - Quadrada   & 15               & 1               & 2             &                                      \\
    \hline
    C2 - Triângular & 4                & 0               & 3             &                                      \\
    \hline
    C3 - Senoidal   & 1                & 0,5             & 2,5           &     \\ \hline                                
    \end{tabular}
    \caption{Formas de onda.}
    \label{tab:form_wave}
    \end{table}

Dica: resolva para um valor $f$ de frequência e depois substitua cada um dos valores correspondentes no resultado. Lembre-se que $\omega = 2\pi f$, com $\omega$ medido em rad/s e $f$ medido em Hz. Observe e comente: qual a relação entre a frequência da senoide e seu valor eficaz?