\documentclass[10pt]{article}
\usepackage[portuguese]{babel}
\usepackage[a4paper, top = 0.8in, bottom = 0.9in, left =0.6in, right = 0.6in]{geometry}
\usepackage{inputenc}
\usepackage{fancyhdr}
\usepackage{multirow}
\usepackage{graphicx}
\usepackage{times}
\usepackage{subcaption}
\usepackage{booktabs}
\usepackage{microtype}
\usepackage{wrapfig}
\usepackage{circuitikz}
\usepackage{titlesec}
\usepackage{wrapfig}
\usepackage{listings}
\usepackage{tikz}

\usepackage{caption}
\usepackage{ragged2e}
\usepackage{adjustbox}
\usepackage{float}
\usepackage{amsmath, mathtools, amssymb}
\usepackage{parskip}
\usepackage{array, makecell}
\usepackage{booktabs}
\titleformat{\section}
{\normalfont\large\bfseries}{\thesection)}{1em}{}
\titleformat{\subsection}
{\normalfont\normalsize\bfseries}{\thesubsection)}{1em}{}
\titleformat{\subsubsection}
{\normalfont\normalsize\bfseries}{\thesubsubsection)}{1em}{}


%\pretitle{\begin{center}\Huge\bfseries}

\pagestyle{fancy}
\fancyhead{}
\centering\chead{\includegraphics[width=15cm]{imagens/unb_fga_logo_extenso.jpg}}
\lfoot{Prática de Circuitos 1 - 2020/1}
\rfoot{\textbf{Prof. Dr. Marcus V. Chaffim Costa}} %nome do rodape
\renewcommand{\footrulewidth}{1pt}

\setlength{\parskip}{0.1cm}
\author{Monitoria - 2019/2}
\title{Tutorial 0}


%arrumação das tabelas

\setlength{\tabcolsep}{0.5em} % for the horizontal padding
{\renewcommand{\arraystretch}{1.2}}% for the vertical padding
\captionsetup[table]{name= \bfseries Tabela}
\captionsetup[figure]{name= \bfseries Figura}
\numberwithin{table}{section}
\numberwithin{figure}{section}
\usepackage{array}
\newcolumntype{C}[1]{>{\centering\arraybackslash}m{#1}}

\aboverulesep=0cm
\belowrulesep=0cm

%configuracoes dos codigos em matlab

\usepackage{color} %red, green, blue, yellow, cyan, magenta, black, white
\definecolor{mygreen}{RGB}{28,172,0} % color values Red, Green, Blue
\definecolor{mylilas}{RGB}{170,55,241}

\lstset{language=Matlab,%
    %basicstyle=\color{red},
    breaklines=true,%
    morekeywords={matlab2tikz},
    keywordstyle=\color{blue},%
    morekeywords=[2]{1}, keywordstyle=[2]{\color{black}},
    identifierstyle=\color{black},%
    stringstyle=\color{mylilas},
    commentstyle=\color{mygreen},%
    showstringspaces=false,%without this there will be a symbol in the places where there is a space
    numbers=left,%
    numberstyle={\tiny \color{black}},% size of the numbers
    numbersep=9pt, % this defines how far the numbers are from the text
    emph=[1]{for,end,break},emphstyle=[1]\color{red}, %some words to emphasise
    %emph=[2]{word1,word2}, emphstyle=[2]{style},    
}

\allowdisplaybreaks


%==============macros para uso nos documentos===================%

\newcommand{\ohm}{\Omega}


\newcommand{\figuras}[4]{
    \begin{figure}[H]
        \centering
        \includegraphics[width={#1}\textwidth]{#2}
        \caption{#3}
        \label{fig:#4}
    \end{figure}
}

\newcommand{\figura}[2]{
    \begin{figure}[H]
        \centering
        \includegraphics[width=#1\textwidth]{#2}
    \end{figure}
}

\newcommand{\subfiguras}[5]{
    \centering
    \begin{subfigure}[H]{#1\textwidth}
        \centering
        \includegraphics[width={#2}\textwidth]{{#3}}
        \caption{{#4}}
        \label{fig:{#5}}
    \end{subfigure}
}

\newcommand{\circuito}[3]{
    \begin{figure}[H]
        \centering
        \begin{circuitikz}[line width=.5pt]
            #1
        \end{circuitikz}
        \caption{{#2}}
        \label{circ:{#3}}    
    \end{figure}
}

\newcommand{\calc}[1]{
    \begin{gather*}
        #1
    \end{gather*}
}





%=====================================================%

\usetikzlibrary{automata}

\begin{document}

\begin{center}
\vspace*{.03cm}
\Large\textbf{Experimento 01:}\\ %titulo
\Large{Instrumentos de Bancada e Geração de Sinais AC}
\end{center}
\justify

\section{Objetivos}

Nesta experiência prosseguimos com a investigação dos demais instrumentos de bancada do laboratório a fim de compreender
seu funcionamento. Então, produziremos sinais AC, a partir dos quais aferiremos parâmetros, com o intuito de aprender a operar
adequadamente os equipamentos e desenvolver conceitos de amplitude, frequência, valor médio e valor eficaz de um sinal.

\section{Estudo pré-laboratorial}

\subsection{Utilizando as Leis de Kirchhoff, resolva os circuitos A e B (Figuras \ref{circ:circ_a} e \ref{circ:circ_b}. Você deverá determinar as tensões elétricas
nos pontos indicados em função das fontes e dos valores de resistores. Obtenha ainda as correntes em $\mathbf{R_1}$ e $\mathbf{R_4}$ para cada caso.}

\begin{figure}[H]
    \centering
    \begin{subfigure}[H]{.4\textwidth}
        \centering
        \begin{circuitikz}[line width=.5pt, american currents, scale = .8, transform shape]
            \draw 

            (0,0) node [ground] {} to [I, l = $I_1$, -*] (0,2) node [above] {A}
            to [R, l = $R_1$] (3,2) node [above] {B} to [R, l = $R_2$, *-] (3,0) to [short, *-*] (0,0)
            ;
            \draw 
            (3,0) to [R, l = $R_4$] (6,0) to [I, l_= $I_2$, *-*] (6,2) node [above] {C}
            to [R, l_= $R_3$] (3,2)
            ;

        \end{circuitikz}
        \caption{Circuito A}
        \label{circ:circ_a} 
    \end{subfigure}
    \begin{subfigure}[H]{.55\textwidth}
        \centering
        \begin{circuitikz}[line width=.5pt, american voltages, scale = .8, transform shape]
            \draw 

            (0,0) node [ground] {} to [V, l = $V_1$, -*, invert] (0,2) node [above] {A}
            to [R, l = $R_1$] (3,2) node [above] {B} to [R, l = $R_2$, *-] (3,0) to [short, *-*] (0,0)
            ;
            \draw 

            (3,2) to [V, invert, l = $V_2$] (5,2) to [R, l = $R_3$] (7,2)  node [above] {C} to [R, l = $R_5$] (10,2) node [above] {D} to [R, l= $R_6$,*-] (10,0) -- (7,0) to [R, l= $R_4$,-*] (7,2)
            ;
            \draw (3,0) to [short, -*] (7,0);
        \end{circuitikz}
        \caption{Circuito B}
        \label{circ:circ_b} 
    \end{subfigure}
    \caption{Circuitos com fontes de corrente e fontes de tensão independentes.}
    \label{fig:circ_b}
\end{figure}




Aplicando a Lei de Kirchhoff para Tensões (LKT) ao circuito A, tem-se
\begin{equation}
    I_3 = I_1 + I_2
\end{equation}

onde $I_3$ é definido como a tensão que sai do nó B e passa pelo $R_2$, como abaixo:

\begin{figure}[H]
    \centering
    \begin{circuitikz}[line width=.5pt, american currents, scale = .8, transform shape]
        \draw 

        (0,0) node [ground] {} to [I, l = I1, -*] (0,2) node [above] {A}
        to [R, l = $R_1$] (3,2) node [above] {B} to [R, l = $R_2$, *-*, v = $I_3$, color = red] (3,0) to [short, *-*] (0,0)
        ;
        \draw 
        (3,0) to [R, l = $R_4$] (6,0) to [I, l_= I2, *-*] (6,2) node [above] {C}
        to [R, l_= $R_3$] (3,2)
        ;

    \end{circuitikz}
\end{figure}

Desse modo, define-se
\begin{align}
    \dfrac{V_A-V_B}{R_1}= I1 && \dfrac{V_C-V_B}{R_3}=I2 && \dfrac{V_B-0}{R_2}=I3 && \dfrac{0-V_D}{R_4}=I2
\end{align}

Isolando as tensões em função das fontes e dos valores dos resistores, encontra-se:

\begin{gather}
    -\dfrac{V_D}{R_4}=I_2 \rightarrow V_D=-I_2R_4\\
    \dfrac{V_B}{R_2}=I_3 \rightarrow V_B = I_3R_2 = (I_1 + I_2) R_2\\
    V_C - V_B = I_2R_3 \rightarrow V_C = I_2R_3 + V_B = I_2R_3 + (I_1 +I_2)R_2\\
    V_A-V_B = I_1R_1 \rightarrow V_A = I_1R_1 + V_B = I_1R_1 +  (I_1+I_2)R_2
\end{gather}

Tem-se, portanto, a seguinte configuração

\begin{align}
    V_A = I_1R_1 + (I_1+I_2)R_2 && V_B = (I_1 + I_2)R_2 && V_C=I_2R_3 + (I_1+I_2)R_2 && V_D = -I_2R_4
\end{align}

As correntes em $R_1$ e $R_4$ são, respectivamente, $I_1$ e $I_2$. 

Utilizando a mesma estratégia de resolução para o circuito B, tem-se

\begin{gather}
    A: V_A = V_1\\
    B: \dfrac{V_B-V_1}{R_1} + \dfrac{V_B}{R_2} + \dfrac{V_B+V_2-V_C}{R_3}=0\\
    C: \dfrac{V_C-V_D}{R_5} + \dfrac{V_C}{R_4} + \dfrac{V_C - V_2 - V_B}{R_3} =0\\
    D: \dfrac{V_D-V_C}{R_5}+ \dfrac{V_D}{R_6}=0
\end{gather}

Isolando as tensões em função das fontes e dos
valores de resistores, encontra-se os seguintes valores:

\begin{gather}
    B: \left[\dfrac{1}{R_1}+\dfrac{1}{R_2}+\dfrac{1}{R_3}\right]V_B + \left[-\dfrac{1}{R_3}\right]V_C = \dfrac{V_1}{R_1}-\dfrac{V_2}{R_3}\\
    C: \left[-\dfrac{1}{R_3}\right]V_B + \left[\dfrac{1}{R_3}+ \dfrac{1}{R_4} + \dfrac{1}{R_5} \right]V_C + \left[\dfrac{-1}{R_5}\right]V_D = \dfrac{V_2}{R_3}\\
    D: \left[-\dfrac{1}{R_5}\right]V_C + \left[\dfrac{1}{R_5}+\dfrac{1}{R_6} \right]V_D = 0
\end{gather}

\subsection{Utilize o método nodal para calcular as tensões e o método dos laços para calcular as correntes em todos os resistores do
circuito C (Figura \ref{circ: circ_c})}

\begin{figure}[H]
    \centering
        \begin{circuitikz}[line width=.5pt, american voltages, scale = .8, transform shape]
            \draw
            (0,0) node [ground] {} to [V, l = $V_1$, invert, *-] (0,4)
            to [R, l = $R_1$, -*] (2,4) -- (4,4) to [R, l = $R_4$] (4,2)
            to [R, l = $R_3$] (4,0) -- (0,0)
            ;
            \draw
            (2,4) to [R, l_= $R_2$] (2,2) to [R, l_= $R_5$, -*] (2,0)
            ; 
            \draw 
            (2,2) to [R, l = $R_6$, *-*] (4,2)
            ;
            \draw (2,4) node [above] {A};
            \draw (2,0) node [below] {B};
        
        \end{circuitikz}    
    \caption{Circuito C}
    \label{circ: circ_c}
\end{figure}

Adicioná-se-a, para facilitação dos cálculos, a inserção de dois nós, C e D, como visto abaixo:

\begin{minipage}{.4\textwidth}
    \begin{figure}[H]
        \centering
            \begin{circuitikz}[line width=.5pt, american voltages, scale = .8, transform shape]
                \draw
                (0,0) node [ground] {} to [V, l = $V_1$, invert, *-] (0,4)
                to [R, l = $R_1$, -*] (2,4) -- (4,4) to [R, l = $R_4$] (4,2)
                to [R, l = $R_3$] (4,0) -- (0,0)
                ;
                \draw
                (2,4) to [R, l_= $R_2$] (2,2) to [R, l_= $R_5$, -*] (2,0)
                ; 
                \draw 
                (2,2) to [R, l = $R_6$, *-*] (4,2)
                ;
                \draw (2,4) node [above] {A};
                \draw (2,0) node [below] {B};
                \draw (2,2) node [left, color = red] {\bfseries C};
                \draw (4,2) node [right, color = red] {\bfseries D};
            
            \end{circuitikz}    
        \caption{Circuito C}
        \label{circ: circ_c_tuto}
    \end{figure}
\end{minipage}
\begin{minipage}{.5\textwidth}
\begin{gather}
    B: V_B=0\\
    A: \dfrac{V_A-V_1}{R_1} + \dfrac{V_A - V_C}{R_2} + \dfrac{V_A-V_D}{R_4}=V_1\\
    C: \dfrac{V_C-V_A}{R_2} + \dfrac{V_C}{R_5} + \dfrac{V_C - V_D}{R_6} = 0\\
    D: \dfrac{V_D - V_A}{R_4} + \dfrac{V_D - V_C}{R_6} + \dfrac{V_D}{R_3}=0
\end{gather}
\end{minipage}



Isolando as tensões em função das fontes e dos valores de resistores, encontra-se os seguintes valores:

\begin{gather*}
    \left[\dfrac{1}{R_1} + \dfrac{1}{R_2} + \dfrac{1}{R_4} \right]V_A + \left[-\dfrac{1}{R_2} \right]V_C + \left[-\dfrac{1}{R_4}\right]V_D = \dfrac{V_1}{R_1}\\
    \left[-\dfrac{1}{R_2}\right]V_A + \left[\dfrac{1}{R_2} + \dfrac{1}{R_5} + \dfrac{1}{R_6}\right]V_C + \left[-\dfrac{1}{R_6}\right]V_D = 0\\
    \left[-\dfrac{1}{R_4}\right]V_A + \left[\dfrac{-1}{R_6}\right]V_C + \left[\dfrac{1}{R_3} + \dfrac{1}{R_4}+\dfrac{1}{R_6}\right]V_D=0
\end{gather*}

Utilizando o método das malhas para calcular as correntes, temos

\begin{minipage}{.4\textwidth}
    \begin{figure}[H]
        \centering
            \begin{circuitikz}[line width=.5pt, american voltages, scale = .8, transform shape]
                \draw
                (0,0) node [ground] {} to [V, l = $V_1$, invert, *-] (0,4)
                to [R, l = $R_1$, -*] (2,4) -- (4,4) to [R, l = $R_4$] (4,2)
                to [R, l = $R_3$] (4,0) -- (0,0)
                ;
                \draw
                (2,4) to [R, l_= $R_2$] (2,2) to [R, l_= $R_5$, -*] (2,0)
                ; 
                \draw 
                (2,2) to [R, l = $R_6$, *-*] (4,2)
                ;
                \draw (2,4) node [above] {A};
                \draw (2,0) node [below] {B};
                \draw (1,2) node[scale=3]{$\circlearrowright$};
                \draw (3,3.1) node[scale=3]{$\circlearrowright$};
                \draw (3,1) node[scale=3]{$\circlearrowright$};
                \draw (1,2) node [color=red] {$I_1$};
                \draw (3,3.1) node [color=red] {$I_2$};
                \draw (3,1) node [color=red] {$I_3$};
            
            \end{circuitikz}    
        \caption{Circuito C método das malhas.}
        \label{circ: circ_c_tuto2}
    \end{figure}
\end{minipage}
\begin{minipage}{.5\textwidth}
    \begin{equation*}
        \begin{cases}
            R_1+(I_1 + I_2)R_2 + (I_1 - I_3)R_5 = 10\\
            R_4 I_2 + (I_2 - I_3)R_6 + (I_2 - I_1)R_2 = 0\\
            R_3 I_3 + (I_3-I_1)R_5 + (I_3 - I_2)R_6 = 0
        \end{cases}
    \end{equation*}

    \begin{equation*}
        \begin{cases}
            [R_1 + R_2 + R_5]I_1 + [- R_2]I_2 + [-R_5]I_3 = 10\\
            [-R_2]I_1+ [R_2 +R_4 +R_6]I_2 + [-R_6]I_3 = 0\\
            [-R_5]I_1 + [-R_6]I_2+ [R_3+R_5+R_6]I_3  
        \end{cases}
    \end{equation*}
    
\end{minipage}



\subsection{Obtenha uma fórmula para a resistência equivalente entre os pontos A e B do circuito C. Dica: retire do circuito a fonte de
alimentação e o resistor R1. Em seguida, utilize uma conversão entre associação delta (triângulo) para estrela.}

Eliminando a fonte e o resistor $R_1$ tem-se a seguinte configuração

\begin{figure}[H]
    \centering
        \begin{circuitikz}[line width=.5pt, american voltages, scale = .8,transform shape]
            \draw
                (0,4) to [short, o-] (4,4) to [R, l = $R_4$] (4,2)
                to [R, l = $R_3$] (4,0) to [short, -o] (0,0)
                ;
                \draw
                (2,4) to [R, l_= $R_2$, *-] (2,2) to [R, l_= $R_5$, -*] (2,0)
                ; 
                \draw 
                (2,2) to [R, l = $R_6$, *-*] (4,2)
                ;
                \draw (0,4) node [left] {A};
                \draw (0,0) node [left] {B};
            
        \end{circuitikz}    
        \caption{Circuito C, exceto o resistor $R_1$ e a fonte de alimentação.}
        \label{circ: circ_c_tuto3}
\end{figure}

Aplicando a tranformação Delta-Estrela, tem-se:

\begin{align}
    R_A=\dfrac{R_2 + R_4}{R_2 \cdot R_4 \cdot R_6} && R_C=\dfrac{R_2 + R_6}{R_2 \cdot R_4 \cdot R_6} && R_D=\dfrac{R_4 + R_6}{R_2 \cdot R_4 \cdot R_6}
\end{align}

Obtendo,

\begin{figure}[H]
    \centering
        \begin{circuitikz}[line width=.5pt, american voltages, scale = .8,transform shape]
            \draw
                (0,4) to [short, o-] (4,4) to [R, l = $R_A$, -*] (4,2)
                to [R, l = $R_D$] (6,2) to [R, l= $R_3$] (6,0) to [short, -o] (0,0)
                ;
                \draw
                (4,2) to [R, l = $R_C$] (2,2) to [R, l_= $R_5$, -*] (2,0)
                ; 
                \draw (0,4) node [left] {A};
                \draw (0,0) node [left] {B};
            
        \end{circuitikz}    
        \caption{Circuito C, exceto o resistor $R_1$ e a fonte de alimentação.}
        \label{circ: circ_c_tuto4}
\end{figure}

Fazendo a associação em série dos resistores inferiores, temos


\begin{minipage}{.5\textwidth}
    \begin{figure}[H]
        \centering
            \begin{circuitikz}[line width=.5pt, american voltages, scale = .8,transform shape]
                \draw
                    (0,4) to [short, o-] (3,4) to [R, l = $R_A$, -*] (3,2)
                    to [R, l = $R_8$] (4,0) to [short, -o] (0,0)
                    ;
                    \draw
                    (3,2) to [R, l_= $R_7$, -*] (2,0)
                    ; 
                    \draw (0,4) node [left] {A};
                    \draw (0,0) node [left] {B};
                
            \end{circuitikz}    
            \caption{Circuito C, exceto o resistor $R_1$ e a fonte de alimentação.}
            \label{circ: circ_c_tuto4}
    \end{figure} 
\end{minipage}
\begin{minipage}{.4\textwidth}
    \begin{equation}
        \begin{gathered}
            R_7 = R_C + R_5\\
            R_8 = R_D + R_3
        \end{gathered}   
    \end{equation}
\end{minipage}


Fazendo-se uma associação em paralelo entre $R_7 \, \text{e} \, R_8$, tem-se

\begin{minipage}{.4\textwidth}
    \begin{figure}[H]
        \centering
            \begin{circuitikz}[line width=.5pt, american voltages, scale = .8,transform shape]
                \draw
                    (0,4) to [short, o-] (3,4) to [R, l = $R_A$, -*] (3,2)
                    to [R, l = $R_9$] (3,0) to [short, -o] (0,0)
                    ; 
                    \draw (0,4) node [left] {A};
                    \draw (0,0) node [left] {B};
                
            \end{circuitikz}    
            \caption{Circuito C, exceto o resistor $R_1$ e a fonte de alimentação.}
            \label{circ: circ_c_tuto4}
    \end{figure}
\end{minipage}
\begin{minipage}{.5\textwidth}
    \begin{gather}
        R_9 = \dfrac{R_7 + R_8}{R_7 \cdot R_8}
    \end{gather}
    
\end{minipage}

que, por uma simples associação em série, culmina em:

\begin{minipage}{.5\textwidth}
    \begin{figure}[H]
        \centering
            \begin{circuitikz}[line width=.5pt, american voltages, scale = .8,transform shape]
                \draw
                    (0,4) to [short, o-] (3,4) to [R, l = $R_{eq}$] (3,0) to [short, -o] (0,0)
                    ; 
                    \draw (0,4) node [left] {A};
                    \draw (0,0) node [left] {B};
                
            \end{circuitikz}    
            \caption{Circuito C, exceto o resistor $R_1$ e a fonte de alimentação.}
            \label{circ: circ_c_tuto4}
    \end{figure}
\end{minipage}
\begin{minipage}{.4\textwidth}
    \begin{gather}
        R_{eq}= R_A + R_9
    \end{gather}
\end{minipage}

\subsection{Fazendo $\mathbf{R_2 = R_3}$ e $\mathbf{R_4 = R_5}$ no circuito C, determine quais modificações deveriam ser feitas no layout do circuito, sem
modificar os valores dos componentes utilizados, para que a tensão se anule sobre o resistor $\mathbf{R_6}$ (ou seja, para obter uma
configuração análoga a uma Ponte de Wheatstone). Dica: faça $\mathbf{i_{R6}}$ igual a zero em suas equações e verifique a relação que
surge entre os resistores restantes.}

Observando a relação final obtida pelo método dos nós e eliminando as correntes que passam pelo resistor $R_6$, tem-se


\begin{gather*}
    \left[-\dfrac{1}{R_2}\right]V_A + \left[\dfrac{1}{R_2} + \dfrac{1}{R_5} + \cancel{\dfrac{1}{R_6}}\right]V_C + \cancel{\left[-\dfrac{1}{R_6}\right]}V_D = 0\\
    \left[-\dfrac{1}{R_4}\right]V_A + \left[\cancel{\dfrac{-1}{R_6}}\right]V_C + \left[\dfrac{1}{R_3} + \dfrac{1}{R_4}+\cancel{\dfrac{1}{R_6}}\right]V_D=0
\end{gather*}

Facilmente, pode-se perceber que os nós C e D são bastante similares diferenciando-se apenas em relação aos resistores. A passagem de corrente depende da diferença de potencial elétrico entre dois pontos de um circuito. Logo, para que não haja passagem de corrente no resistor $R_6$, necessita-se que $V_C$ e $V_D$ sejam iguais. 

Portanto, apenas permutar a posição de $R_3$ e $R_4$ faz com que a corrente sobre $R_6$ seja zero.

\subsection{Simule os circuitos A, B e C (Figuras 2.1a, 2.1b e 2.2) e obtenha os valores correspondentes de tensão e corrente usando o
QUCS 0.0.19. Inclua no estudo pré-laboratorial os desenhos do circuito simulado juntamente com as medições realizadas. Para
os circuitos das Figuras 2.1a e 2.1b, assuma $\mathbf{R_1 = R_4 = 2,2k\ohm}$, $\mathbf{R_2 = R_6 = 1k\ohm}$, $\mathbf{R_3 = R_5 = 4,7k\ohm}$, $\mathbf{V_1 = 12V}$, $\mathbf{V_2 = 20V}$,
$\mathbf{I_1 = 12mA}$ e $\mathbf{I_2 = 20mA}$. Para o circuito da Figura 2.2, assuma $\mathbf{R_1 = 2,2k\ohm}$, $\mathbf{R_2 = R_3 = 1k\ohm}$, $\mathbf{R_4 = R_5 = 4,7k\ohm}$, $\mathbf{R_6 =
100\ohm}$ e $\mathbf{V_1 = 10V}$. Em seguida, substitua os mesmos valores nas fórmulas encontradas nos itens 2.1 e 2.2. Complete as
tabelas a seguir com seus resultados teóricos e simulados.}

Assumiu-se $R_1 = R_4 = 2,2k\ohm$, $R_2 = 1k\ohm$, $R_3 = 4,7k\ohm$, $I_1 = 12mA$ e $I_2 = 20mA$. Em seguida, substituiu-se os mesmos valores nas fórmulas encontradas para verificar os
cálculos:

\begin{gather*}
    V_A = 12mA\cdot 2,2k\ohm + (12mA + 20 mA)\cdot 1k\ohm = 58,4V\\
    V_B = (12mA + 20mA)1k\ohm = 32V\\
    V_C = 12mA\cdot 4,7k\ohm + (12mA + 20mA)\cdot 1k\ohm = 126V\\
    V_D = -20mA \cdot 2,2 k\ohm = -44V
\end{gather*}

Portanto
\begin{itemize}
\item Abra um novo esquemático
\end{itemize}

\figuras{.3}{imagens/nth_th/new_sch.png}{Abra um novo esquemático.}{new_schm}

\begin{itemize}
  \item Na aba componentes, vá em componenetes agrupados e coloque cinco resistores no esquemático. Vá em fontes e coloque duas fontes de tensão DC. Vá em ponteiras e coloque uma ponteira de corrente e uma de tensão.
\end{itemize}

\begin{figure}[H]
  \subfiguras{.3}{.6}{imagens/nth_th/ins_res.png}{}{}
  \subfiguras{.3}{.6}{imagens/nth_th/ins_dc.png}{}{}
  \subfiguras{.3}{.6}{imagens/nth_th/probes.png}{}{}
  \caption{Insira os componentes no esquemático}
  \label{fig:comp_th_nt}
\end{figure}

\begin{itemize}
  \item Conecte os componentes sem esquecer da referência do terra e ajuste seus valores para os pedidos no exercício.
\end{itemize}

\figuras{.7}{imagens/nth_th/circuito_a.png}{Circuito exigido pelo exercício.}{circ_exe}

\begin{itemize}
  \item Coloque a simulação DC no esquemático, salve e simule.
\end{itemize}

\figura{.3}{imagens/nth_th/ins_dc_sim.png}
\figuras{1}{imagens/nth_th/save_sim.png}{Insira a simulação depois salve e simule.}{ins_dc_sim}

\begin{itemize}
  \item Vá em diagramas e insira uma tabela. Coloque o valor da corrente $i_{R3}.I$ e da tensão $V_{AB}.V$.
\end{itemize}

\figuras{.4}{imagens/nth_th/table_thev.png}{Insira uma tabela para obter os valores simulados.}{ins_table}

\begin{itemize}
  \item Assim, verifica-se os valores pedidos no exercício.
\end{itemize}

\figuras{.7}{imagens/nth_th/final_circuito_a.png}{Circuito exigido pelo exercício.}{circ_a_final}

\begin{itemize}
  \item Vá em $Arquivo\rightarrow Salvar \, como...$ e mude o nome do arquivo para usar o circuito já montado na próxima simulação.
\end{itemize}

\figuras{.3}{imagens/nth_th/save_as.png}{Salve a simulação.}{save_as}

\begin{itemize}
  \item Desative a fonte $V_1$ como circuito fechado utilizando a ferramenta Desativar/Ativar e clicando duas vezes sobre a fonte, até a marcação ficar verde.
\end{itemize}

\figura{.5}{imagens/nth_th/dis_en.png}
\figuras{.7}{imagens/nth_th/final_circuito_a_dis1.png}{Desabilite a fonte de tensão V1.}{dis_v1}

\begin{itemize}
  \item Vá em $Arquivo\rightarrow Salvar \, como...$ e mude o nome do arquivo para usar o circuito já montado na próxima simulação.
\end{itemize}

\figuras{.25}{imagens/nth_th/save_as.png}{Salve o projeto.}{save_as_2}

\begin{itemize}
  \item Ative a fonte $V_1$ e desative a fonte $V_2$.
\end{itemize}

\figura{.5}{imagens/nth_th/dis_en.png}
\figuras{.7}{imagens/nth_th/final_circuito_a_dis2.png}{Desabilite a fonte de tensão V2.}{dis_v2}

\begin{itemize}

  \item Vá em $Arquivo\rightarrow Salvar \, como...$ e mude o nome do arquivo para usar o circuito já montado na próxima simulação.
\end{itemize}

\figuras{.25}{imagens/nth_th/save_as.png}{Salve o projeto.}{save_as_3}

\begin{itemize}
  \item Ative $V_2$ e insira um ponteira de corrente para medir a corrente sobre o resistor $R_L$.
\end{itemize}

\figuras{.7}{imagens/nth_th/final_circuito_a_dis3.png}{Habilite ambas as fontes e coloque o amperímetro para a medição do $i_{RL}$}{circ_a_3}

\begin{itemize}
  \item Encontre a tensão de circuito aberto $V_{oc}$ para o circuito. Para isso, remova a carga e meça a corrente de curto-circuito $i_{sc}$.
\end{itemize}

\figuras{.8}{imagens/nth_th/final_isc_voc.png}{Retire o $R_L$ e meça a tensão de circuito aberto.}{voc}

\begin{itemize}

  \item Vá em $Arquivo\rightarrow Salvar \, como...$ e mude o nome do arquivo para usar o circuito já montado na próxima simulação.
\end{itemize}

\figuras{.25}{imagens/nth_th/save_as.png}{Salve o projeto.}{save_as_4}

\newpage

\begin{itemize}
  \item Insira as equações para encontrar o $R_{eq}$ e o $i_{sc}$..
\end{itemize}

\figuras{.8}{imagens/nth_th/final_req.png}{Resultado exigido pelo exercício.}{final_req}



\begin{itemize}
  \item Montou-se o circuito equivalente de Thévenin utilizando uma fonte de tensão em série com um resistor $R_{Th}$ e o resistor de carga. Ajustou-se a fonte de acordo com a tensão de Thévenin medida e o resistor R Th de acordo com a resistência equivalente medida.

  \item Montou-se o circuito equivalente de Norton utilizando uma fonte de corrente em paralelo com um resistor $R_{No}$ e o resistor de carga. Ajustou-se a fonte de acordo com a corrente de Norton medida e o resistor $R_No$ de acordo com a resistência equivalente medida.

  \item Vá em $Arquivo\rightarrow Salvar \, como...$ e mude o nome do arquivo para usar o circuito já montado na próxima simulação.
\end{itemize}


\figuras{.8}{imagens/nth_th/final_eq_th_nt.png}{Resultado equivalente Thévenin e Norton.}{final_nth}

\section{Experimento}

\subsection{Geração e medição de ondas}

\textbf{a)} Ajuste o gerador de funções para gerar cada uma das formas de onda indicadas na tabela \ref{tab:form_wave}, visualizando-as no
osciloscópio. Certifique-se de que o gerador de funções está ajustado no modo “alta impedância” (explique o que essa
opção faz) e que o osciloscópio está ajustado para “acomplamento DC” (Sim, DC! Explique o motivo), com o ganho do
probe em 1x. Verifique se o trigger do osciloscópio está associado ao CH1.

Para cada curva, meça com o multímetro e com o osciloscópio os valores de tensão AC (eficaz) e DC (médio). Compare
os valores medidos com os dois instrumentos e justifique.

\noindent \textbf{b)} Utilizando a última curva ajustada (C3), altere a frequência para os seguintes valores: 1 Hz, 10 Hz, 100 Hz, 1 kHz, 10
kHz, 50 kHz, 100 kHz, 250 kHz, 500 kHz, 1 MHz, medindo novamente com o multímetro e o osciloscópio os valores DC
e AC da tensão para cada frequência. O que mudou? Este resultado faz sentido teoricamente? Os valores medidos
correspondem aos calculados no seu estudo pré-laboratorial? Explique em termos de limitação de medida do multímetro
para frequências muito altas e/ou muito baixas.

\noindent \textbf{c)} Assim como a fonte de alimentação DC, o gerador de funções também possui resistência interna. Monte o circuito da
Figura \ref{circ:{circ_imp_ger_func}}  e meça a tensão de saída em CH1. Se $R_{in}$ = 0, qual seria o valor esperado de $V_{CH1}$? Ao invés disso, quanto foi
observado? Com base nesta informação, e utilizando o conceito de divisão de tensão, estime a resistência interna do gerador de funções. Em seus cálculos, utilize o valor real do resistor R.

\circuito{

    \draw 

    (0,0) node [ground] {} to [short, *-] (-2,0)
    to [sI, l=$V_f$] (-2,1) to [R, l=$R_{in}$] (-2,3) to [short, -*] (2,3) node [above] {CH1} to [R, l=$R{=}100 \ohm$] (2,0) to [short] (0,0)
    ;
    \draw 

    [dashed] (-1,-.5) to [short] (-1,3.5) to [short] (-3,3.5) to [short] (-3,-.5) -- (-1,-.5);

    \draw 

    (-3.5, 2) node [left,rotate = 90] {Gerador real}
    ;

}{Circuito para estimação de $R_{in}$ do gerador de funções.}{circ_imp_ger_func}

\noindent \textbf{d)} As pontas de prova de um osciloscópio deveriam ter resistência de entrada infinita, mas na prática possuem Rin grande e
finita. Monte o o circuito da Figura \ref{circ:{circ_imp_osc}}  e meça a tensão de saída em CH1. Se $R_{in} \rightarrow \infty$, qual seria o valor esperado de $V_{CH1}$? Ao
invés disso,quanto foi observado? Com base nesta informação,e utilizando o conceito de divisão de tensão,estime a resistência
interna da ponta de prova. Em seus cálculos, utilize os valores reais de $R_1$ e $R_2$.

\circuito{
    \draw 

    (0,0) node [ground] {} to [sI, l = $V_f$] (0,3) to [R, l= $R_1{=} 1.2 M\ohm$, -*] (3,3)
    to [R, l = $R_{in}$] (3,0) to [short, *-] (0,0)
    ;
    \draw 
    (3,0) to [short] (5,0) to [R, l_= $R_2{=}1.2M\ohm$] (5,3) -- (3,3)
    ;

    \draw 
    [dashed] (2.5,-.5) -- (4,-.5) -- (4,3.5) -- (2.5,3.5) -- (2.5,-.5)
    ;
    \draw (3,-.6) node [below] {Ponta de prova real};

}{Circuito para estimação de $R_{in}$ da ponta de prova do osciloscópio.}{circ_imp_osc}

\subsection{Funções matemáticas do osciloscópio}

Gere uma onda senoidal com uma frequência de 1kHz, $2V_{pp}$ (tensão pico-a-pico) e $0V_m$ (tensão média) e alimente o CH1 do
osciloscópio com essa forma de onda. Utilize a fonte de alimentação para gerar uma tensão DC de 2 V e alimente o CH2 do
osciloscópio. Verifique o resultado das operações soma “+”, subtração “-” e multiplicação “*” usando o botão MATH do osciloscópio.

\subsection{Espectro de frequência de uma forma de onda}

Gere uma onda senoidal com uma frequência de 100kHz e alimente o CH1 do osciloscópio com essa forma de onda. Anote os
valores de frequência e amplitude da onda. Utilizando a função FFT do botão MATH do osciloscópio gere o espectro da função
senoidal criada e esboce-o. Varie a frequência para 50kHz e 10kHz, descrevendo e justificando o que acontece.

\end{document}