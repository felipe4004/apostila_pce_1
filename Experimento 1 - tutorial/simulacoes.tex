\section{Simulações}

Usando o simulador de circuitos QUCS 0.0.19, faça a simulação do procedimento experimental descrito no item 2.4e. Além dos desenhos do circuito, inclua em seu estudo pré-laboratorial as formas de onda de cada canal, sobrepostas à forma de onda da fonte.


\subsection{Onda Quadrada}

\begin{itemize}
    \item Abra um novo esquemático.
\end{itemize}

\figuras{.4}{imagens/sim/form_waves/new_sch}{Insersão de um novo esquemático.}{new_sch}

\begin{itemize}
    \item Na aba Componentes, vá em componentes
    agrupados e coloque um resistor no esquemático.
    Vá em fontes e coloque um pulso de tensão no
    esquemático. Vá em ponteiras e coloque um
    voltímetro no esquemático.
\end{itemize}

\begin{figure}[H]
    \centering
    \subfiguras{.3}{.8}{imagens/sim/form_waves/res_ins}{}{}
    \subfiguras{.3}{.8}{imagens/sim/form_waves/volt_prove}{}{}
    \subfiguras{.3}{.8}{imagens/sim/form_waves/source_pulse}{}{}
    \caption{Inserção dos componentes.}
    \label{comp_ins}
\end{figure}

\newpage

\begin{itemize}
    \item Como queremos uma onda quadrada e o offset da função é 1 V, U1 deve ser -1 V e U2 deve ser 3 V. $T_r$ e $T_f$ devem ser igual a zero. O sinal se inicia em zero então T1 deve ser 0. Como a frequência é de 15 kHz, encontra-se o valor do período de aproximadamente $66,67 \mu s$, logo T2 deve ser $33.33 \mu s$ (arredonda-se para $33,5 \mu s$ ).
\end{itemize}

\figuras{.5}{imagens/sim/form_waves/param_pulse}{Parâmetro do pulso.}{new_sch}

\begin{itemize}
    \item Será utilizada a simulação transiente para se
    observar o comportamento do circuito. Essa
    simulação realiza uma análise temporal que permitirá
    observar gráficos em relação ao tempo.
\end{itemize}

\figuras{.3}{imagens/sim/form_waves/trans_sim}{Inserção da simulação transiente.}{sim_trans}

\begin{itemize}
    \item Como o período da onda é de 67ns, coloque tempo
    suficiente para visualizar um período completo e
    resolução grande o suficiente para gerar a onda.
\end{itemize}

\figuras{.5}{imagens/sim/form_waves/param_transient_pulse}{Parâmetros para a simulação transiente.}{trans_sim_param}

\begin{itemize}
    \item Utilize um resistor de $1 k\ohm$ para verificar a onda gerada. Conecte os componentes sem esquecer a referência do terra como na figura abaixo:
\end{itemize}

\figuras{.5}{imagens/sim/form_waves/square_circ}{Circuito a ser montado.}{circ_square}

\begin{itemize}
    \item Salve e simule. Vá em Diagramas e insira um plano cartesiano.
\end{itemize}


\figura{1}{imagens/sim/form_waves/save_sim}
\figuras{.3}{imagens/sim/form_waves/cartesian_diagram}{Inserção de um plano cartesiano.}{}

\begin{itemize}
    \item Coloque a tensão medida pelo voltímetro no gráfico
    e arrume o limite para aparecer o pulso da onda.
\end{itemize}

\figuras{.5}{imagens/sim/form_waves/param_diagram}{Parâmetros para o diagrama cartesiano.}{}

\newpage

\begin{itemize}
    \item Assim, verifica-se a forma de onda pedida no
    exercício.
\end{itemize}

\figuras{.8}{imagens/sim/form_waves/final_square}{Resultado exigido pelo exercício.}{final1}

\subsection{Onda Triangular}

\begin{itemize}
    \item Vá em Arquivo $\rightarrow$ Salvar como... e mude o nome do
    arquivo para utilizar o esquemático já montado para
    a segunda onda.

    \item Como o offset da função é 0 V, U1 deve ser -3 V e U2
    deve ser 3 V. O sinal se inicia em zero então T1 deve
    ser 0. Como a frequência é de 4 kHz, encontra-se o
    valor do período de 0,25 ms, logo T2 deve ser $250 \mu s$.
    Como queremos uma onda triangular, $T_r$ e $T_f$ devem
    ser metade do período, ou seja, $125\mu s$.
\end{itemize}

\figuras{.5}{imagens/sim/form_waves/param_trian_source}{Parâmetros para a fonte de tensão triangular.}{param_trian_source}

\begin{itemize}
    \item Será utilizada a simulação transiente para se
    observar o comportamento do circuito. Essa
    simulação realiza uma análise temporal que permitirá
    observar gráficos em relação ao tempo.
\end{itemize}


\figuras{.3}{imagens/sim/form_waves/trans_sim}{Inserção da simulação transiente.}{sim_trans}

\begin{itemize}
    \item Como o período da onda é de $250 \mu s$, coloque
    tempo suficiente para visualizar um período
    completo e resolução grande o suficiente para gerar
    a onda, ou seja, maior ou igual a $250 \mu s$.
\end{itemize}

\figuras{.4}{imagens/sim/form_waves/param_transient_trian}{Configuração dos parâmetros para a simulação transiente.}{param_transient_trian}

\begin{itemize}
    \item Salve e simule o arquivo. Vá em Diagramas e insira um plano cartesiano
\end{itemize}

\figura{1}{imagens/sim/form_waves/save_sim}
\figuras{.3}{imagens/sim/form_waves/cartesian_diagram}{Inserção de um plano cartesiano.}{}

\begin{itemize}
    \item Coloque a tensão medida pelo voltímetro no gráfico.
\end{itemize}

\figuras{.5}{imagens/sim/form_waves/param_diagram}{Parâmetros para o diagrama cartesiano.}{}

\newpage

\begin{itemize}
    \item Assim, verifica-se a forma de onda pedida no
    exercício.
\end{itemize}

\figuras{.8}{imagens/sim/form_waves/final_trian}{Resultado exigido pelo exercício.}{final2}

\subsection{Onda Senoidal}

\begin{itemize}
    \item Vá em Arquivo $\rightarrow$ Salvar como... e mude o nome do
    arquivo para utilizar o esquemático já montado para
    a terceira onda.

    \item Como se quer uma onda senoidal, troque a fonte
    para uma fonte de tensão AC.
\end{itemize}

\figuras{.3}{imagens/sim/form_waves/ac_source}{Inserção de uma fonte AC.}{ac_source}

\begin{itemize}
    \item Configure sem o offset.
\end{itemize}

\figuras{.5}{imagens/sim/form_waves/param_sin_source}{Parâmetros da fonte AC.}{param_ac_source}

\begin{itemize}
    \item Utilize a ferramenta de Equação para colocar o
    offset na onda.
\end{itemize}

\figura{.7}{imagens/sim/form_waves/ins_eqn}
\figuras{.5}{imagens/sim/form_waves/eqn_param}{Parâmetros da equação de offset.}{eqn_param}

\begin{itemize}
    \item Será utilizada a simulação transiente para se
    observar o comportamento do circuito. Essa
    simulação realiza uma análise temporal que permitirá
    observar gráficos em relação ao tempo.
\end{itemize}

\figuras{.3}{imagens/sim/form_waves/trans_sim}{Inserção da simulação transiente.}{sim_trans}

\begin{itemize}
    \item Como o período da onda é de 1 ms, coloque tempo
    suficiente para visualizar um período completo e
    resolução grande o suficiente para gerar a onda.
\end{itemize}


\figuras{.5}{imagens/sim/form_waves/param_transient_sin}{Parâmetros para a simulação transiente.}{sim_trans}

\begin{itemize}
    \item Salve e simule o arquivo. Vá em Diagramas e insira um plano cartesiano.
\end{itemize}

\figuras{.3}{imagens/sim/form_waves/cartesian_diagram}{Inserção de um plano cartesiano.}{digram_ins}

\begin{itemize}
    \item Coloque a equação calculada no gráfico.
\end{itemize}

\figuras{.5}{imagens/sim/form_waves/param_diagram_sin}{Parâmetro do diagrama cartesiano.}{param_diagram}

\begin{itemize}
    \item Assim, verifica-se a forma de onda pedida no
    exercício.
\end{itemize}

\figuras{.9}{imagens/sim/form_waves/final_sin}{Resultado esperado pelo exercício.}{final3}

\newpage

\subsection{Extra: Verificação no Octave/Matlab}

No Octave ou Matlab, digite o escreva o código abaixo e execute.

\lstinputlisting{imagens/sim/octave/wave_forms.m}
 
\figuras{1}{imagens/sim/octave/waves_form}{Formas de onda obtidas.}{waves_form}