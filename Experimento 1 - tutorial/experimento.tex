\section{Experimento}

\subsection{Geração e medição de ondas}

\textbf{a)} Ajuste o gerador de funções para gerar cada uma das formas de onda indicadas na tabela \ref{tab:form_wave}, visualizando-as no
osciloscópio. Certifique-se de que o gerador de funções está ajustado no modo “alta impedância” (explique o que essa
opção faz) e que o osciloscópio está ajustado para “acomplamento DC” (Sim, DC! Explique o motivo), com o ganho do
probe em 1x. Verifique se o trigger do osciloscópio está associado ao CH1.

Para cada curva, meça com o multímetro e com o osciloscópio os valores de tensão AC (eficaz) e DC (médio). Compare
os valores medidos com os dois instrumentos e justifique.

\noindent \textbf{b)} Utilizando a última curva ajustada (C3), altere a frequência para os seguintes valores: 1 Hz, 10 Hz, 100 Hz, 1 kHz, 10
kHz, 50 kHz, 100 kHz, 250 kHz, 500 kHz, 1 MHz, medindo novamente com o multímetro e o osciloscópio os valores DC
e AC da tensão para cada frequência. O que mudou? Este resultado faz sentido teoricamente? Os valores medidos
correspondem aos calculados no seu estudo pré-laboratorial? Explique em termos de limitação de medida do multímetro
para frequências muito altas e/ou muito baixas.

\noindent \textbf{c)} Assim como a fonte de alimentação DC, o gerador de funções também possui resistência interna. Monte o circuito da
Figura \ref{circ:{circ_imp_ger_func}}  e meça a tensão de saída em CH1. Se $R_{in}$ = 0, qual seria o valor esperado de $V_{CH1}$? Ao invés disso, quanto foi
observado? Com base nesta informação, e utilizando o conceito de divisão de tensão, estime a resistência interna do gerador de funções. Em seus cálculos, utilize o valor real do resistor R.

\circuito{

    \draw 

    (0,0) node [ground] {} to [short, *-] (-2,0)
    to [sI, l=$V_f$] (-2,1) to [R, l=$R_{in}$] (-2,3) to [short, -*] (2,3) node [above] {CH1} to [R, l=$R{=}100 \ohm$] (2,0) to [short] (0,0)
    ;
    \draw 

    [dashed] (-1,-.5) to [short] (-1,3.5) to [short] (-3,3.5) to [short] (-3,-.5) -- (-1,-.5);

    \draw 

    (-3.5, 2) node [left,rotate = 90] {Gerador real}
    ;

}{Circuito para estimação de $R_{in}$ do gerador de funções.}{circ_imp_ger_func}

\noindent \textbf{d)} As pontas de prova de um osciloscópio deveriam ter resistência de entrada infinita, mas na prática possuem Rin grande e
finita. Monte o o circuito da Figura \ref{circ:{circ_imp_osc}}  e meça a tensão de saída em CH1. Se $R_{in} \rightarrow \infty$, qual seria o valor esperado de $V_{CH1}$? Ao
invés disso,quanto foi observado? Com base nesta informação,e utilizando o conceito de divisão de tensão,estime a resistência
interna da ponta de prova. Em seus cálculos, utilize os valores reais de $R_1$ e $R_2$.

\circuito{
    \draw 

    (0,0) node [ground] {} to [sI, l = $V_f$] (0,3) to [R, l= $R_1{=} 1.2 M\ohm$, -*] (3,3)
    to [R, l = $R_{in}$] (3,0) to [short, *-] (0,0)
    ;
    \draw 
    (3,0) to [short] (5,0) to [R, l_= $R_2{=}1.2M\ohm$] (5,3) -- (3,3)
    ;

    \draw 
    [dashed] (2.5,-.5) -- (4,-.5) -- (4,3.5) -- (2.5,3.5) -- (2.5,-.5)
    ;
    \draw (3,-.6) node [below] {Ponta de prova real};

}{Circuito para estimação de $R_{in}$ da ponta de prova do osciloscópio.}{circ_imp_osc}

\subsection{Funções matemáticas do osciloscópio}

Gere uma onda senoidal com uma frequência de 1kHz, $2V_{pp}$ (tensão pico-a-pico) e $0V_m$ (tensão média) e alimente o CH1 do
osciloscópio com essa forma de onda. Utilize a fonte de alimentação para gerar uma tensão DC de 2 V e alimente o CH2 do
osciloscópio. Verifique o resultado das operações soma “+”, subtração “-” e multiplicação “*” usando o botão MATH do osciloscópio.

\subsection{Espectro de frequência de uma forma de onda}

Gere uma onda senoidal com uma frequência de 100kHz e alimente o CH1 do osciloscópio com essa forma de onda. Anote os
valores de frequência e amplitude da onda. Utilizando a função FFT do botão MATH do osciloscópio gere o espectro da função
senoidal criada e esboce-o. Varie a frequência para 50kHz e 10kHz, descrevendo e justificando o que acontece.