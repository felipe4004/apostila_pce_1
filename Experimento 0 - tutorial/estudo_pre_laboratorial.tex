\section{Objetivos}
Nesta experiência investigaremos alguns dos instrumentos de bancada do laboratório a fim de compreender como
funcionam. Em seguida produziremos sinais DC e realizaremos medições de tensão,corrente e resistência. O intuito é desenvolver habilidades na manipulação dos equipamentos.

\section{Estudo pré-laboratorial}

\subsection{Instrumentos de Bancada}

\subsubsection{Fonte de alimentação}

\begin{itemize}
    \item \textbf{O que é faixa de tensão de saída do equipamento?}
        
        É a faixa de tensões em que o aparelho fornece em sua saída. Ela pode ser limitada nominalmente pelo usuário através de um potenciômetro (em algumas literaturas, \textit{knob}). A faixa de tensões vária em função do modo de operação escolhido entre os canais de saída da fonte.

    \item \textbf{O que é faixa de corrente de saída do equipamento?}

        É a faixa de corrente que o aparelho fornece em sua saída. É análogo à faixa de tensões, mas conta, vale ressaltar, com o ajuste de corrente limite. A faixa de corrente também vária em função do modo de operação configurada na fonte.
    \item \textbf{O que é corrente limite?}

        É a corrente máxima configurada pelo usuário a partir da qual a proteção contra sobrecorrente atua. A quantidade de corrente opera a partir da resistência oferecida pelo circuito, ou seja, se há algum componente que esteja danificado, há probabilidades que haja o sobreaquecimento e, com isso, até a combustão do componente.
\end{itemize}

\subsubsection{Modos de Operação de Fontes de Alimentação}

Atualmente há, majoritariamente, dois tipos de fonte no laboratório: MPL-1303 e MPL-3305, ambas da Minipa e, portanto, compartilham do mesmo manual. A principal diferença reside entre a quantidade de canais de saída existentes: a MPL-1303 possui apenas um canal de saída, enquanto a MPL-3305 possui dois canais de saída. Implica-se, então, que a MPL-1303 possui apenas o modo de operação simples.

Para todas os modos de operação, devemos configurar as teclas de Seleção do Modo de Conexão (também conhecidas como Teclas de Tracking). Elas mudam a conexão entre as fontes de acordo com as associações desejadas. As configurações de Tracking dos modos Independente, Série ou paralelo para a fonte MPL-3305 estão elencados abaixo:

\figuras{.5}{imagens/instr_banc/tracking}{Configurações de Conexão}{tracking}

\begin{itemize}
     
    \item \textbf{Fixa}

        A fonte MPL-3305 conta com três saídas, sendo duas variáveis e uma fixa de 5V.
        
        \newpage

    \item \textbf{Simples}

        É o modo onde os canais de saída operam independentemente. A corrente máxima é de 5A.

        \figuras{.5}{imagens/instr_banc/op_simples}{Modo de Operação Simples.}{op_simples}
    \item \textbf{Paralelo}

        Nessa condição, os canais operam paralelamente e, consequentemente, as corrente se somam. A corrente máxima é, portanto, 10A.

        \figuras{.4}{imagens/instr_banc/op_paralelo}{Modo de Operação Paralela.}{op_paralelo}

    \item \textbf{Série}

        Nessa condição, os canais operam em serialmente e, consequentemente, as tensões se somam. A tensão máxima é 64V.

        \figuras{.4}{imagens/instr_banc/op_serie}{Modo de Operação Série.}{op_serie}
    \item \textbf{Simétrica}
    
        Nesta condição, pode-se conseguir um terra comum para ambos os canais variáveis, com saída positiva e negativa de, no máximo, +32V e -32V, respectivamente. 
    
        \figuras{.4}{imagens/instr_banc/op_simetric}{Modo de Operação Simétrica.}{op_simetrica}
\end{itemize}

O ajuste de corrente limite é ilustrado na figura abaixo:

\figuras{.4}{imagens/instr_banc/corr_lim}{Configuração da Corrente Limite na fonte MPL-3305}{}

Observe que, propositalmente, não houve um detalhamento das peculiaridades de cada um dos modos de operação. Através dos manuais dos equipamentos, procure compreender o modo de funcionamento das fontes de alimentação. Como guia, responda as seguintes perguntas:

a) Fonte de alimentação modelo MPL-1303 Minipa:

\begin{itemize}
    \item Qual a faixa de tensão de saída desse equipamento?
    \item Qual a faixa de corrente de saída desse equipamento?
    \item Como definir um valor máximo de corrente de saída para a fonte (corrente limite)?
\end{itemize}

\vspace{.5em}

b) Fonte de alimentação modelo MPL-3305 Minipa:

\begin{itemize}
    \item Qual a faixa de tensão de saída para esse equipamento?
    \item Qual a faixa de corrente de saída para esse equipamento?
    \item Quais os limites de tensão e corrente obtidos em cada um dos modos de operação?
\end{itemize}

\subsubsection{Multímetro}

\begin{itemize}
    \item \textbf{Medição de Tensão com Multímetro:}Posicione a chave rotativa para medição de voltagem contínua e conecte as pontas de prova em paralelo no circuito em teste.
    
    \figuras{.2}{imagens/instr_banc/mult_tens}{Medição de tensão com o multímetro.}{tens_mult}

    \item \textbf{Medição de corrente com o Multímetro:} Posicione a chave rotativa na maior escala possível de corrente e ajuste para uma visualização adequada. Conecte as pontas de prova em série no local a ser medido.
    
    \figuras{.35}{imagens/instr_banc/corr_mult}{Medição de corrente com o multímetro.}{corr_mult}
    
    \item \textbf{Medição de Resistência com o multímetro:} Posicione a chave rotativa em "$\ohm$". Conecte as pontas de prova sobre o objeto a ser medido.
    
    \figuras{.2}{imagens/instr_banc/res_mult}{Medição de Resistência com o multímetro.}{res_mult}

    \item \textbf{Integridade de Trilhas:} Posicione a chave rotativa para medição de continuidade. O aparelho sonoriza um \textit{beep} quando a resistência de um circuito for menor que $10\ohm$.
    
    \figuras{.2}{imagens/instr_banc/int_trilh}{Teste de integridade de trilhas com o multímetro.}{int_trilh}
\end{itemize}

Assim como as fontes de tensão, há dois modelos de multímetro disponíveis no laboratório atualmente: ET-1110 da Minipa e o TOOZ DT830 Series. Estude a manipulação e as características nos manuais do aparelho.



