\newpage
\section{Experimento}

\subsection{Multímetro}

\noindent a) Meça com o multímetro o valor dos resistores disponíveis em cima da bancada. Confira o valor nominal indicado pelo código
de cores e calcule o erro percentual. Verifique se ele se encontra dentro da tolerância especificada pelo fabricante do resistor.

\subsection{Geração de tensões}

\noindent a) Use a fonte de alimentação e escolha o modo de operação mais apropriado para obter os seguintes sinais de tensão DC:

\begin{itemize}
    \item +10 V e -10 V c/ terra comum (máx 3 A);
    \item 40 V (máx 3 A);
    \item 5 V (máx 6 A);
    \item 5 V (máx 3 A);
    \item 10 V (máx 3 A);
\end{itemize}

Verifique os valores obtidos com o auxílio do multímetro e comente sobre o resultado.

\noindent b) Monte o circuito da Figura 11, utilizando o resistor $R = 100 \ohm$ e ajuste a fonte de alimentação em 3 V, 5 V e 10 V. Meça a tensão e a corrente sobre o resistor R em cada caso. Discuta os valores observados e compare-os com os obtidos nos cálculos teóricos da sessão 3.1, justificando os valores observados.

\subsection{Modo tensão/corrente constante}

\noindent a) Ajuste a fonte de alimentação para fornecer uma tensão de 10 V e corrente máxima de 30 mA.

b) Monte o circuito da Figura 12, com $R = 100 \ohm$ e substituindo o resistor R2 pelo potenciômetro ajustado para os seguintes valores:

\begin{itemize}
    \item $1 k\ohm$;
    \item $500 \ohm$;
    \item $200 \ohm$;
    \item $100 \ohm$;
    \item $50 \ohm$.
\end{itemize}


Em seguida, verifique com o multímetro os valores de tensão $V_f$ e corrente $i_f$ fornecidos pela fonte para diferentes valores de $R_2$.
Explique o que você observou e justifique o comportamento da fonte comparando com os cálculos teóricos de $V_f$ e $i_f$ realizados
nas simulações.