\section{Impedâncias}

\subsection{Circuito A}
A figura \ref{fig:circuit_a_imp} retrata o modelo de impedância de entrada para o circuito Amplificador Operacional Inversor.

\begin{figure}[H] 
    \centering
    \begin{adjustbox}{width=.5\textwidth}
    \begin{circuitikz}[line width=.5pt]
        \draw
        
        (0,0) node [ground] {} to [american controlled voltage source , l_= $A(V^+-V^-)$, invert] (0,3)
            to [R = $R_o$, -o] (5,3) node [right] {$V_o$}
            (4,3) to [short, *-] (4,5)
            to [R=$R_2$, f<=$i_2$] (-4,5)
            to [short, -*] (-4,3)
            to [R = $R_1$, f<=$i_1$, -o] (-7,3)
            (-7.7,3) node[right] {$V_1$}
            
            (-4,3) to [short, f=$i_i$] (-2,3)
            to [R = $R_i$] (-2,1)
            to [short] (-4,1)
            to [short] (-4,0) 
            to [short,-*] (0,0)
            to [short, -*] (3,0)
            
        ;
            
    \end{circuitikz}
    \end{adjustbox}
    \caption{Circuito A}
    \label{fig:circuit_a_imp}
\end{figure}

Podemos aplicar a Lei de Kirchhoff no nó de entrada para observar o comportamento da resistência de entrada:

    

\begin{gather*}
    i_1 =i_2+i_i\\\\
    \dfrac{V_- - V_1}{R_1} = \dfrac{V_o - V_-}{R_2} + \dfrac{V_+ - V_-}{R_i}
    \Longrightarrow
    \dfrac{V_+ - V_-}{R_i} = \dfrac{ V_- - V_1}{R_1} - \dfrac{V_o - V_-}{R_2}
    \\
    \\
    \dfrac{V_+ - V_-}{R_i} = \dfrac{(V_- - V_1)R_2 - (V_o - V_-)R_1}{R_1 R_2}\\
\end{gather*}
\begin{equation}
    R_i = \dfrac{(V_+ - V_-) R_1 R_2}{(V_- - V_1) R_2 - (V_o - V_-)R_1}
\end{equation}

Para a resistência de saída, aplicamos a Lei de Kirchhoff no nó de saída, donde obtemos

\begin{gather*}
    \dfrac{V_o - V_-}{R_2} = \dfrac{A(V_+ - V_-) - V_o}{R_o}
\end{gather*}
\begin{equation}
        R_o = \dfrac{(AV_+ - AV_- - V_o)R_2}{V_o - V_-}
\end{equation}

\subsection{Circuito B}

    O circuito da figura \ref{fig:circuit_b_imp} retrata o modelo de circuito Amplificador Operacional Somador Inverso.

\begin{figure}[H]
    \centering
    \begin{adjustbox}{width=.5\textwidth}
    \begin{circuitikz}[line width = .5pt]
        \draw
            (0,0) node [ground] {} to [american controlled voltage source , l_= $A(V^+-V^-)$, invert] (0,3)
            to [R = $R_o$, -o] (5,3) node [right] {$V_o$}
            (4,3) to [short, *-] (4,5)
            to [R=$R_2$, f<^=$i$] (-4,5)
            to [short, -*] (-4,3) -- (-5,3)
            to [R = $R_3$, f<=$i_2$, -o] (-8,3)
            node[left] {$V_2$}
            (-5,3) to [short,*-] (-5,4)
            to [R = $R_1$, f<=$i_2$, -o] (-8,4) node [left] {$V_1$}
            (-5,3) to [short] (-5,2)
            to [R = $R_4$, f<=$i_3$, -o] (-8,2) node [left] {$V_3$}
            
            (-4,3) to [short, f=$i_i$] (-2,3)
            to [R = $R_i$] (-2,1)
            to [short] (-4,1)
            to [short] (-4,0) 
            to [short,-*] (0,0)
            to [short, -*] (3,0)
            
        ;
    \end{circuitikz}
    \end{adjustbox}
    \caption{Circuito B}
    \label{fig:circuit_b_imp}
\end{figure}

    Podemos observar o comportamento da impedância de entrada analisando o nó de entrada com a Lei de Kirchhoff:
    
    \begin{gather*}
        i + i_i = i_1 + i_2 + i_3\\ \\
        \dfrac{V_o - V_-}{R_2} + \dfrac{V_+ - V_-}{R_i} = \dfrac{V_- - V_1}{R_1} + \dfrac{V_- - V_2}{R_3} + \dfrac{V_- - V_3}{R_4}\\
        \dfrac{V_+ - V_-}{R_i} = \dfrac{V_- - V_1}{R_1} + \dfrac{V_- -V_2}{R_3} + \dfrac{V_- -V_3}{R_4} - \dfrac{V_o - V_-}{R_2}\\
        \dfrac{V_+ - V_-}{R_i} = \dfrac{(V_- - V_1)R_2 R_3 R_4 - (V_o - V_-)R_1 R_3 R_4 + (V_- - V_2)R_1 R_2 R_4 + (V_- - V_3)R_1 R_2 R_3}{R_1 R_2 R_3 R_4}\\
        R_i=\dfrac{R_1 R_2 R_3 R_4(V_+ - V_-)}{(V_- - V_1)R_2 R_3 R_4 - (V_o - V_-)R_1 R_3 R_4 + (V_- - V_2)R_1 R_2 R_4 + (V_- - V_3)R_1 R_2 R_3}
    \end{gather*}
    
    Podemos analisar a impedância de saída a partir do nó de saída do circuito, donde temos
    \begin{gather*}
        \dfrac{V_o - V_-}{R_2} = \dfrac{A(V_+ - V_-) - V_o}{R_o}\\
        R_o = \dfrac{(AV_+ - AV_- - V_o)R_2}{V_o - V_-}
    \end{gather*}

\subsection{Circuito C}

\begin{figure}[H]
    \centering
    \begin{adjustbox}{width=.5\textwidth}
    \begin{circuitikz}[line width = .5pt]
        \draw
            (0,0) node [ground] {} to [american controlled voltage source , l_= $A(V^+-V^-)$, invert] (0,3)
            to [R = $R_o$, -o] (5,3) node [right] {$V_o$}
            (4,3) to [short, *-] (4,6)
            to [R=$R_2$, f<^=$i_2$] (-4,6)
            to [short, -*] (-4,4.5)
            to [R = $R_1$, f<=$i_1$, -o] (-8,4.5)
            node[left] {$V_1$}
            (-8,2.5) node [left] {$V_2$}
            to [R = $R_3$, f>^=$i_3$, o-*] (-4,2.5)
            
            (-4,4.5) to [short, f=$i_i$] (-2,4.5)
            to [R = $R_i$] (-2,2.5)
            to [short] (-4,2.5)
            to [R = $R_4$, f_= $i_4$](-4,0)
            to [short,-*] (0,0)
            to [short, -*] (3,0)
            
        ;
    \end{circuitikz}
    \end{adjustbox}
    \caption{Circuito C}
    \label{fig:circuit_c_imp}
\end{figure}

 O circuito da figura \ref{fig:circuit_c_imp} é uma abstração do funcionamento do amplificador operacional arranjado na configuração de Subtrator.
 
 Analisando o nó de entrada, podemos obter a impedância de entrada. Desse modo, 
 
\begin{gather*}
     i_i = i_2 - i_1\\ \\
     \dfrac{V_+ - V_-}{R_i} = \dfrac{V_o - V_-}{R_2} - \dfrac{V_- - V_1}{R_1}
\end{gather*}
\begin{equation}
         R_i = \dfrac{(V_+ - V_-)R_1 R_2}{(V_o - V_-)R_1 - (V_- - V_1)R_2}
\end{equation}
\begin{center}
    ou
\end{center}
\begin{gather*}
     i_i = i_4 - i_3\\
     \\
     \dfrac{V_+-V_-}{R_i} = \dfrac{V_o - V_-}{R_4} - \dfrac{V_- - V_2}{R_3}\\
\end{gather*}
\begin{equation}
         R_i = \dfrac{(V_+ - V_-)R_3 R_4}{V_+ R_3 - (V_+ - V_2)R_4}
\end{equation}

 Agora, analisando o nó de saída, temos:
\begin{gather*}
    \dfrac{V_o - V_-}{R_2} = \dfrac{A(V_+ - V_-) - V_o}{R_o}\\
    R_o = \dfrac{(AV_+ - AV_- -V_o)R_2}{V_o - V_-}
\end{gather*}
