\documentclass[10pt]{article}
\usepackage[portuguese]{babel}
\usepackage[a4paper, top = 0.8in, bottom = 0.9in, left =0.6in, right = 0.6in]{geometry}
\usepackage[utf8]{inputenc}
\usepackage{fancyhdr}
\usepackage{multirow}
\usepackage{graphicx}
\usepackage{times}
\usepackage{subcaption}
\usepackage{microtype}
\usepackage{wrapfig}
\usepackage{circuitikz}
\usepackage{titlesec}
\usepackage{wrapfig}
\usepackage{ragged2e}
\usepackage{adjustbox}
\usepackage{float}
\usepackage{amsmath, mathtools}
\usepackage{parskip}
\usepackage{array, makecell}
\usepackage{booktabs}
\titleformat{\section}
{\normalfont\large\bfseries}{\thesection)}{1em}{}
\titleformat{\subsection}
{\normalfont\normalsize\bfseries}{\thesubsection)}{1em}{}
\titleformat{\subsubsection}
{\normalfont\normalsize\bfseries}{\thesubsubsection)}{1em}{}


%\pretitle{\begin{center}\Huge\bfseries}

\pagestyle{fancy}
\fancyhead{}
\centering\chead{\includegraphics[width=15cm]{imagens/tumbnail_unb.jpg}}
\lfoot{Prática de Circuitos 1 - 2019/1}
\rfoot{\textbf{Prof. Dr. Marcus V. Chaffim Costa}} %nome do rodape
\renewcommand{\footrulewidth}{1pt}

\setlength{\parskip}{0.1cm}
\author{Monitoria - 2019/2}
\title{Protocolo experimental 06}

\begin{document}

\begin{center}
\vspace*{.2cm}
\Large\textbf{TE 5:}\\ %titulo
\Large{Críterios de correção}
\end{center}
\justify

\section{Divisão de notas}
A nota do TE foi dividida igualmente entre os experimentos. Assim, cada experimento tem um valor fixo de 0 até 5. Esse critério é o mais razoável possível, realizando uma distribuição justa de nota.
\subsection{Experimento 9 - Circuito com amplificador operacional}
A nota teve distribuição por igual entre os procedimentos, a saber: 1,667 por item. Este foi dividido novamente igualmente entre os itens dos procedimentos. Listando, temos:
\begin{enumerate}
	\item \textbf{Procedimento 3.1:} 32 itens, logo 0,05 por item.
	\item \textbf{Procedimento 3.2:} 7 itens, logo 0,23 por item.
	\item \textbf{Procedimento 3.3:} 26 itens, logo 0,065 por item.
\end{enumerate}

A nota final é resultado da soma simples dos pontos obtidos em cada procedimento.

\subsection{Experimento 10 - Circuito Integrador e Diferenciador}
Copiosamente, o mesmo roteiro aplicou-se ao experimento 10. Em listagem,

\begin{enumerate}
	\item \textbf{Procedimento 3.1:} 20 itens, logo 0,125 por item.
	\item \textbf{Procedimento 3.2:} 24 itens, logo 0,105 por item.
\end{enumerate}

A nota final é resultado da soma simples dos pontos obtidos em cada procedimento.

\section{Critério de arredondamento}
O arredondamento se deu apenas para valores com 3 casas decimais, para cima ou para baixo, dependendo em que posição referente a linha média (acima ou abaixo de 0,005) se encontra o valor.

\section{Parâmetro para a correção dos itens}
Os itens tiveram correção segundo a própria natureza. Se um valor teórico, então seu valor exato, ou pelo menos próximo ao de um erro de arredondamento. Se um valor experimental, a veracidade do valor segundo a análise com o valor ideal. Se um valor que depende tanto de valores experimentais quanto de teóricos, então analisou-se o cálculo realizado per si.

\section{Avaliação das simulações}
	As simulações foram caracterizadas ao início com nota máxima, cujo desconto ocorre a cada falta encontrada. Em lista,
	
	\begin{enumerate}
		\item \textbf{Valores incorretos:} desconto de 0,5 pontos.
		\item \textbf{Não simulado:} desconto de 0,5 pontos.
		\item \textbf{Sem gráfico ou gráfico incorreto:} desconto de 1 ponto.
		\item \textbf{Circuito incorreto:} desconto de 1 ponto.
		\item \textbf{Falta de simulações:} desconto de 0,5 por simulação não feita.
		
		Cuida-se para que observe os arquivos enviados pelos alunos. Eles, por vezes, estão corretos, mas apresentam algum erro ou têm valores de simulações não definidos. Cabe análise a cada caso.
	\end{enumerate}
	
\section{Avaliação dos protocolos}
	
	Pela natureza subjetiva dos protocolos, não analisou-se as faltas de informações ou possíveis erros de alinhamento, gramática, entre outros. O critério recaiu sobre o esmero observado na confecção do protocolo. Os alunos que colocaram pouca ou nenhuma informação, tiveram desconto de pontos. Os que não prezaram pela organização do texto, também foram descontados. O desconto ocorreu de forma branda, pois cada aluno elabora o protocolo de maneira a encontrar mais dinâmica com ele. Entretanto, cabe ressaltar ainda o caráter acadêmico dos protocolos.

\end{document}