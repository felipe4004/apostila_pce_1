\newpage
\begin{center}
    \Large \bfseries 119148 – Prática de Circuitos Eletrônicos 1 – Folha de Dados
\end{center}

\vspace{.5cm}

\setlength{\parindent}{0.0cm}

\makebox[.3\textwidth][l]{Turma:\enspace\rule{2cm}{0.4pt}}
\makebox[.3\textwidth][l]{Bancada:\enspace\rule{2cm}{.4pt}}
\makebox[.3\textwidth][l]{Data:\enspace\rule{1cm}{.4pt}/\rule{1cm}{.4pt}/\rule{1cm}{.4pt}}
\\
\\
\makebox[.7\textwidth][l]{Nome:\enspace\rule{.6\textwidth}{0.4pt}}
\makebox[.5\textwidth][l]{Matrícula:\enspace\rule{1cm}{.4pt}/\rule{2cm}{.4 pt}}

\vspace*{5mm}
\begin{center}
    \large \bfseries Experimento 09: Circuitos com Amplificador Operacional
\end{center}
\vspace{5mm}

Resistores usados:
\\
\\
\begin{center}
$R_{1}=$\enspace \rule{1.8cm}{.4pt}\enspace$\pm$\enspace\rule{1.8cm}{.4pt}\enspace$[\Omega]$\hspace{1cm}$R_{2}=$\enspace \rule{1.8cm}{.4pt}\enspace$\pm$\enspace \rule{1.8cm}{.4pt}\enspace$[\Omega]$
\par\end{center}
\begin{center}
$R_{3}=$\enspace \rule{1.8cm}{.4pt}\enspace$\pm$\enspace\rule{1.8cm}{.4pt}\enspace$[\Omega]$\hspace{1cm}$R_{4}=$\enspace\rule{1.8cm}{.4pt}\enspace$\pm$\enspace \rule{1.8cm}{.4pt}\enspace$[\Omega]$
\par\end{center}

\vspace{5mm}

\noindent Procedimento 6.1: Caracterização - Tabela 1

\begin{table}[H]
\begin{tabular}{|C{2cm}|>{\centering}p{5cm}|>{\centering}p{2cm}|>{\centering}p{2cm}|>{\centering}p{2cm}|>{\centering}p{2cm}|}
\hline 
\multirow{2}{*}{Circuito} & {\small{}Configuração das Entradas}{\small\par}

Senoide com $f=$\enspace\rule{1cm}{.4pt}\enspace Hz & Saída 

$V_{o_{pp}}$ & Ganho Experim. & Ganho Teórico & Ganho \\ \%Erro\tabularnewline
\toprule 
\multirow{1}{*}{A} & $V_{1}=0,5V_{pp}$ &  &  &  & \tabularnewline
\midrule
 & $V_{1}=V_{2}=V_{3}=0,5V_{pp}$ &  &  &  & \tabularnewline
\cmidrule{2-6}
B & $V_{1}=V_{2}=0,5V_{pp}$ e $V_{3}=0V_{pp}$ &  &  &  & \tabularnewline
\cmidrule{2-6}
 & $V_{1}=0,5V_{pp}$ e $V_{2}=V_{3}=0V_{pp}$ &  &  &  & \tabularnewline
\midrule
 & $V_{1}=V_{2}=0,5V_{pp}$ &  &  &  & \tabularnewline
\cmidrule{2-6}
C & $V_{1}=0,5V_{pp}$ e $V_{2}=0V_{pp}$ &  &  &  & \tabularnewline
\cmidrule{2-6}
 & $V_{1}=0V_{pp}$ e $V_{2}=0,5V_{pp}$ &  &  &  & \tabularnewline
\bottomrule
\end{tabular}
\caption*{\textbf{Tabela 1} - Avaliação das características de circuitos com amplificador operacional}
\end{table}


Procedimento 6.2 a): Slew-rate
\\
\\
Entrada: Onda quadrada com amplitude $V=$\enspace\rule{1.3cm}{.4pt}
$V_{pp}$ e $f=$\enspace\rule{1.2cm}{.4pt} \enspace$\textrm{Hz}$\\
\\
$\frac{\delta V(t)}{dt}=$\enspace \rule{1.5cm}{.4pt}\enspace $\frac{V}{\mu s}$
\vspace{0.5cm}

Procedimento 3.2 b): Saturação
\\
\\
Entrada: Onda senoidal com amplitude $V=$\enspace\rule{1.4cm}{.4pt}
$V_{pp}$ e $f=$\enspace \rule{1.5cm}{.4pt}\enspace $\textrm{Hz}$\\
\\
\\
$V_{o}(t)_{m\acute{a}x}=$\enspace\rule{1.5cm}{.4pt}\enspace $V$\hspace{3cm}$V_{o}(t)_{m\acute{\imath}n}=$\enspace\rule{1.5cm}{.4pt}\enspace$V$\\
\vspace{0.5cm}

\pagebreak

Procedimento 6.3): Resposta em frequência - Tabela 2


\begin{table}[H]
\centering
\begin{tabular}{|>{\centering}p{5cm}|>{\centering}p{2cm}|>{\centering}p{2cm}|}
\hline 
Senoide com $V_{1}(t)=$\enspace\rule{1.5cm}{.4pt}\enspace $V_{pp}$ & Saída 

$V_{o_{pp}}$ & Ganho Experimental\tabularnewline
\hline 
$10\,\textrm{Hz}$ &  & \tabularnewline
\hline 
$100\,\textrm{Hz}$ &  & \tabularnewline
\hline 
$500\,\textrm{Hz}$ &  & \tabularnewline
\hline 
$1\,\textrm{kHz}$ &  & \tabularnewline
\hline 
$10\,\textrm{kHz}$ &  & \tabularnewline
\hline 
$20\,\textrm{kHz}$ &  & \tabularnewline
\hline 
$30\,\textrm{kHz}$ &  & \tabularnewline
\hline 
$40\,\textrm{kHz}$ &  & \tabularnewline
\hline 
$50\,\textrm{kHz}$ &  & \tabularnewline
\hline 
$75\,\textrm{kHz}$ &  & \tabularnewline
\hline 
$100\,\textrm{kHz}$ &  & \tabularnewline
\hline 
$500\,\textrm{kHz}$ &  & \tabularnewline
\hline 
$1\,\textrm{MHz}$ &  & \tabularnewline
\hline 
$5\,\textrm{MHz}$ &  & \tabularnewline
\hline 
$10\,\textrm{MHz}$ &  & \tabularnewline
\hline 
$20\,\textrm{MHz}$ &  & \tabularnewline
\hline 
\end{tabular}

\caption{\textbf{Tabela 2} - Resposta em frequência do amplificador operacional.}
\end{table}