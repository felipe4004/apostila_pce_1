\section{Experimento}

\subsection{Caracterização de circuitos com resistores e Amplificador
Operacional}

Monte cada um dos circuitos das figuras A, B e C, com tensões de alimentação
$V_{ss}=-10\textrm{V}$ e $V_{dd}=+10\textrm{V}$. Utilize os valores
definidos pelo professor para $R_{1}$, $R_{2}$, $R_{3}$ e $R_{4}$.
Estabeleça experimentalmente a relação entre as amplitudes pico-a-pico
dos sinais de entrada $V_{1}(t)$, $V_{2}(t)$ e $V_{3}(t)$ e a amplitude
pico-a-pico do sinal de saída $V_{o}(t)$ observados no osciloscópio.
Use o gerador de funções nas entradas, com sinais senoidais de amplitudes
definidas segundo a Tabela 1 da Folha de Dados, e com a frequência
arbitrada pelo professor. 

Meça com precisão os valores das resistências $R_{1}$, $R_{2}$,
$R_{3}$ e $R_{4}$ com um multímetro e substitua estes valores nas
expressões obtidas no item 2.1. Compare os resultados experimentais
com os valores teóricos para o ganho de tensão de cada circuito e
de cada configuração mostrados na Tabela 1 da Folha de Dados.

\subsection{Efeitos não-lineares}

\textbf{a) Slew-rate:} Determine a maior taxa de variação da tensão
por unidade de tempo $(\delta V(t)/dt)$ na saída do circuito A. Utilize
uma entrada quadrada com grande amplitude e frequência. 

\noindent\textbf{b) Saturação:} Verifique qual é a amplitude máxima de excursão
da tensão de saída $V_{o}(t)$ do circuito A. Que fatores limitam
na prática a excursão da tensão de saída? Use uma grande amplitude
de entrada, em baixa-frequência $(f<1\,\textrm{kHz})$.

\subsection{Resposta em frequência}

Usando a configuração do circuito A, aumente gradativamente a frequência
do sinal de entrada até o limite do gerador (circuito A) e anote os
valores correspondentes de ganho. Explique o comportamento do ganho
em função da frequência. Use uma entrada com pequena amplitude, para
que não ocorra influência significativa do slew-rate.
