\newpage

\section{Procedimento Experimental}

\subsection{Superposição de Sinais}

\noindent a) Monte o circuito da Fig. 2.3, com $R_1 = 100 \ohm$ ,$ R_2 = 4, 7 k\ohm$, $R_3 = R_4 = 1 k\ohm$ , $R_L = 2, 2 k\ohm$ , $V_1 = 3 V$ e $V_2 = 2 V$ . Com as duas fontes de tensão ligadas, use o multímetro para medir a tensão no resistor $R_L$ e a corrente no resistor $R_3$. 

\noindent b) Coloque $V_1$ em repouso e, mantendo $V_2$ ligada, meça a tensão no resistor $R_L$ e a corrente no resistor $R_3$. 

\noindent c) Agora, coloque $V_2$ em repouso e, mantendo $V_1$ ligada, meça a tensão no resistor $R_L$ e a corrente no resistor $R_3$. A partir dos resultados obtidos, discuta: o princípio da superposição foi verificado na tensão do resistor $R_L$? E na corrente do resistor $R_3$? Explique.


\subsection{Circuitos equivalentes Thévenin e Norton}


\noindent a) Monte o circuito da Fig. 2.3, usando $R_1 = 100 \ohm$ , $R_2 = 4, 7 k\ohm$ , $R_3 = R_4 = 1 k\ohm$ , $V 1 = 3 V$ e $V 2 = 2 V$ . Use um resistor de carga $R_L = 2, 2 k\ohm$ . Com um multímetro, meça a tensão entre os terminais A e B, bem como a corrente na carga $R_L $.      

\noindent c)Encontre a tensão de circuito aberto $V_oc$ para o circuito. Para isso remova a carga e meça a tensão entre os pontos A e B com o multímetro. A seguir, meça a corrente de curto-circuito $i_sc$. O que estes valores de tensão e de corrente representam? 

\noindent c) Com ambas as fontes em repouso, meça a resistência equivalente de Thévenin $R_eq$ entre os terminais A e B (em aberto e sem carga). Calcule a corrente de curto-circuito esperada a partir de $V_oc$ e $R_eq$. O que este valor de resistência e de corrente representam? 

\noindent d) Monte o circuito equivalente de Thévenin utilizando uma fonte de tensão em série com um potenciômetro e o resistor de carga. Ajuste a fonte de acordo com a tensão de Thévenin medida e o potenciômetro de acordo com a resistência equivalente medida. Observe e registre a tensão e a corrente na carga $R_L$ no novo circuito montado. Os resultados foram os esperados? Explique.
