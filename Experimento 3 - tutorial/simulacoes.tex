\begin{itemize}
\item Abra um novo esquemático
\end{itemize}

\figuras{.3}{imagens/nth_th/new_sch.png}{Abra um novo esquemático.}{new_schm}

\begin{itemize}
  \item Na aba componentes, vá em componenetes agrupados e coloque cinco resistores no esquemático. Vá em fontes e coloque duas fontes de tensão DC. Vá em ponteiras e coloque uma ponteira de corrente e uma de tensão.
\end{itemize}

\begin{figure}[H]
  \subfiguras{.3}{.6}{imagens/nth_th/ins_res.png}{}{}
  \subfiguras{.3}{.6}{imagens/nth_th/ins_dc.png}{}{}
  \subfiguras{.3}{.6}{imagens/nth_th/probes.png}{}{}
  \caption{Insira os componentes no esquemático}
  \label{fig:comp_th_nt}
\end{figure}

\begin{itemize}
  \item Conecte os componentes sem esquecer da referência do terra e ajuste seus valores para os pedidos no exercício.
\end{itemize}

\figuras{.7}{imagens/nth_th/circuito_a.png}{Circuito exigido pelo exercício.}{circ_exe}

\begin{itemize}
  \item Coloque a simulação DC no esquemático, salve e simule.
\end{itemize}

\figura{.3}{imagens/nth_th/ins_dc_sim.png}
\figuras{1}{imagens/nth_th/save_sim.png}{Insira a simulação depois salve e simule.}{ins_dc_sim}

\begin{itemize}
  \item Vá em diagramas e insira uma tabela. Coloque o valor da corrente $i_{R3}.I$ e da tensão $V_{AB}.V$.
\end{itemize}

\figuras{.4}{imagens/nth_th/table_thev.png}{Insira uma tabela para obter os valores simulados.}{ins_table}

\begin{itemize}
  \item Assim, verifica-se os valores pedidos no exercício.
\end{itemize}

\figuras{.7}{imagens/nth_th/final_circuito_a.png}{Circuito exigido pelo exercício.}{circ_a_final}

\begin{itemize}
  \item Vá em $Arquivo\rightarrow Salvar \, como...$ e mude o nome do arquivo para usar o circuito já montado na próxima simulação.
\end{itemize}

\figuras{.3}{imagens/nth_th/save_as.png}{Salve a simulação.}{save_as}

\begin{itemize}
  \item Desative a fonte $V_1$ como circuito fechado utilizando a ferramenta Desativar/Ativar e clicando duas vezes sobre a fonte, até a marcação ficar verde.
\end{itemize}

\figura{.5}{imagens/nth_th/dis_en.png}
\figuras{.7}{imagens/nth_th/final_circuito_a_dis1.png}{Desabilite a fonte de tensão V1.}{dis_v1}

\begin{itemize}
  \item Vá em $Arquivo\rightarrow Salvar \, como...$ e mude o nome do arquivo para usar o circuito já montado na próxima simulação.
\end{itemize}

\figuras{.25}{imagens/nth_th/save_as.png}{Salve o projeto.}{save_as_2}

\begin{itemize}
  \item Ative a fonte $V_1$ e desative a fonte $V_2$.
\end{itemize}

\figura{.5}{imagens/nth_th/dis_en.png}
\figuras{.7}{imagens/nth_th/final_circuito_a_dis2.png}{Desabilite a fonte de tensão V2.}{dis_v2}

\begin{itemize}

  \item Vá em $Arquivo\rightarrow Salvar \, como...$ e mude o nome do arquivo para usar o circuito já montado na próxima simulação.
\end{itemize}

\figuras{.25}{imagens/nth_th/save_as.png}{Salve o projeto.}{save_as_3}

\begin{itemize}
  \item Ative $V_2$ e insira um ponteira de corrente para medir a corrente sobre o resistor $R_L$.
\end{itemize}

\figuras{.7}{imagens/nth_th/final_circuito_a_dis3.png}{Habilite ambas as fontes e coloque o amperímetro para a medição do $i_{RL}$}{circ_a_3}

\begin{itemize}
  \item Encontre a tensão de circuito aberto $V_{oc}$ para o circuito. Para isso, remova a carga e meça a corrente de curto-circuito $i_{sc}$.
\end{itemize}

\figuras{.8}{imagens/nth_th/final_isc_voc.png}{Retire o $R_L$ e meça a tensão de circuito aberto.}{voc}

\begin{itemize}

  \item Vá em $Arquivo\rightarrow Salvar \, como...$ e mude o nome do arquivo para usar o circuito já montado na próxima simulação.
\end{itemize}

\figuras{.25}{imagens/nth_th/save_as.png}{Salve o projeto.}{save_as_4}

\newpage

\begin{itemize}
  \item Insira as equações para encontrar o $R_{eq}$ e o $i_{sc}$..
\end{itemize}

\figuras{.8}{imagens/nth_th/final_req.png}{Resultado exigido pelo exercício.}{final_req}



\begin{itemize}
  \item Montou-se o circuito equivalente de Thévenin utilizando uma fonte de tensão em série com um resistor $R_{Th}$ e o resistor de carga. Ajustou-se a fonte de acordo com a tensão de Thévenin medida e o resistor R Th de acordo com a resistência equivalente medida.

  \item Montou-se o circuito equivalente de Norton utilizando uma fonte de corrente em paralelo com um resistor $R_{No}$ e o resistor de carga. Ajustou-se a fonte de acordo com a corrente de Norton medida e o resistor $R_No$ de acordo com a resistência equivalente medida.

  \item Vá em $Arquivo\rightarrow Salvar \, como...$ e mude o nome do arquivo para usar o circuito já montado na próxima simulação.
\end{itemize}


\figuras{.8}{imagens/nth_th/final_eq_th_nt.png}{Resultado equivalente Thévenin e Norton.}{final_nth}
